%% Name macro
\newcommand{\na}[1]{{\sc #1}}

\newcommand{\ic}[1]{\texttt{#1}}

%%%%%%%%%%%%%%% Project Specific Commands %%%%%%%%%%%%%%%
\newcommand{\frontend}{front end}
\newcommand{\Frontend}{Front end}

\newcommand{\backend}{back end}
\newcommand{\Backend}{Back end}

\newcommand{\ui}{user interface}
\newcommand{\Ui}{User interface}

\newcommand{\sql}{SQL}
\newcommand{\sqldb}{\sql-database}

\newcommand{\kin}{Kin}
\newcommand{\Kin}{Kin}

%%%%%%%%%%%%%%%% General Custom Commands %%%%%%%%%%%%%%%%
% Insert figures easily (Uses the file name as the label):
% \fig{SIZE (decimal)}{FILE NAME}{CAPTION}
\newcommand{\fig}[3]{
    \begin{figure}[H] % Alternatively [htbp] 
        \centering
        \includegraphics[width=#1\textwidth]{figures/#2}
        \caption{#3}
        \label{fig:#2} % Note that "fig:" is automatically added to the label here
    \end{figure}
}

% Insert a definition boxes easily with the box command:
% \dbox{TITLE}{box:LABEL}{CONTENT}
\newcommand{\dbox}[3]{
    \vspace*{0.5cm}
    \begin{greybox}[label={#2}]{#1}{Definition~box}
        #3
    \end{greybox}
    \vspace*{0.5cm}
}

% Quote a source with the excerpt command
% \excerpt{CONTENT}{AUTHOR NAME \cite{TAG}}
\newcommand{\excerpt}[2]{
    \begin{quote}
        \textit{#1}
    \end{quote}
    \begin{center}
        #2
    \end{center}
}

%%%%%%%%%%%%%%%%%%% EASY REFERENCING %%%%%%%%%%%%%%%%%%%
% Avoid using \ref{} to make sure you reference in the same way throughout the report

% Figure (Use file name as label reference)
\newcommand{\figref}[1]{figure \ref{fig:#1}} % "fig:" is added automatically here
\newcommand{\Figref}[1]{Figure \ref{fig:#1}} % "fig:" is added automatically here

% Custom warning which is used to warn the user, 
% if 'sec:', 'chap:' ect. is missing from the beginning of the label
\newcommand{\throwWarningIfNotBeginWith}[2]{\IfBeginWith*{#1}{#2}{}{\PackageWarning{macros.tex}{No '#2' found in label '#1'}}}
% Chapter
\newcommand{\chapref}[1]{chapter \ref{#1}\throwWarningIfNotBeginWith{#1}{chap:}}
\newcommand{\Chapref}[1]{Chapter \ref{#1}\throwWarningIfNotBeginWith{#1}{chap:}}
% Section
\newcommand{\secref}[1]{section \ref{#1}\throwWarningIfNotBeginWith{#1}{sec:}}
\newcommand{\Secref}[1]{Section \ref{#1}\throwWarningIfNotBeginWith{#1}{sec:}}
% Subsection
\newcommand{\subsecref}[1]{subsection \ref{#1}\throwWarningIfNotBeginWith{#1}{sec:}}
\newcommand{\Subsecref}[1]{Subsection \ref{#1}\throwWarningIfNotBeginWith{#1}{sec:}}
% Appendix
\newcommand{\appref}[1]{appendix \ref{#1}\throwWarningIfNotBeginWith{#1}{app:}}
\newcommand{\Appref}[1]{Appendix \ref{#1}\throwWarningIfNotBeginWith{#1}{app:}}
% Definition Box
\newcommand{\dboxref}[1]{definition box \ref{#1}\throwWarningIfNotBeginWith{#1}{dbox:}}
\newcommand{\Dboxref}[1]{Definition box \ref{#1}\throwWarningIfNotBeginWith{#1}{dbox:}}
% Table
\newcommand{\tabref}[1]{table \ref{#1}\throwWarningIfNotBeginWith{#1}{tab:}}
\newcommand{\Tabref}[1]{Table \ref{#1}\throwWarningIfNotBeginWith{#1}{tab:}}
% Code snippet
\newcommand{\snipref}[1]{code snippet \ref{#1}\throwWarningIfNotBeginWith{#1}{snip:}}
\newcommand{\Snipref}[1]{Code snippet \ref{#1}\throwWarningIfNotBeginWith{#1}{snip:}}
% Use cases
\newcommand{\ucref}[1]{use case \ref{#1}\throwWarningIfNotBeginWith{#1}{uc:}}
\newcommand{\Ucref}[1]{Use case \ref{#1}\throwWarningIfNotBeginWith{#1}{uc:}}
% Equations
% the \eqref command is occupied by the amsmath package,
% so we need to hack a bit to hijack the command
\let\amsmatheqref\eqref %save eqref definition in \amsmatheqref variable
\renewcommand{\eqref}[1]{equation \amsmatheqref{#1}}
\newcommand{\Eqref}[1]{Equation \amsmatheqref{#1}}
