Machine learning has many applications. 
Since its discovery in the 1950s, machine learning \fxwarning{It is very general, you can directly mention the application of deep learning models. For instance, "the growing use of these type models..."} has been used to solve complex problems in various domains \fxwarning{such as...} \fxnote{This is very broad. Other articles starts closer to their domain. This is either done by introducing the application domain (construction industry) or time series forecasting on accelerometer data (the reasoning or beginning behind. Also if none or insufficient research has been done it is mentioned in the beginning)}\fxnote{Needs citation}. \fxwarning{Then, mention that you want to address the problem using the previous idea (machine/deep learning) for this specific setting, and why it will be useful}
In the 1990s the focus of machine learning shifted from a knowledge-driven approach to a data-driven approach \cite{marr2016short}\fxnote{See first fxnote}. \fxwarning{How this change affect your setting? Is a data-driven approach better?}
This is when scientists started using machine learning to analyze large amounts of data and draw conclusions from the results \cite{marr2016short}. 
One of the problems with this, is that it takes a lot of computing power to process data at this scale. \fxwarning{It is not clear why.}
In present time, this problem is amplified by the fact, that Python is the programming language \fxwarning{It goes to very specific details, it can be more general. For example, "high-level languages that are large and complex"}, which has the best support for machine learning algorithms, through its large amount of libraries and user friendly syntax\fxnote{Needs citation. Seems like a holdning}. \fxwarning{Until here, the idea seems like: data-driven requires a lot of computation + Python is big, but the problem definition is not explicit. Try to detail it before talking about the low-consumption devices }
This creates a problem when it comes to low consumption devices. 
If you have a low consumption device, which has low computing power and limited battery capacity, running machine learning inference can be a demanding task on the device's hardware \fxwarning{Here, you can make the constrast to the problem definition, why something big does not run properly in these devices?} \fxnote{This should be the first sentence with some rewriting imo. Quick introduction to low consumption devices and their limitations - hence our }. 
This can be combatted by using simple algorithms, \fxwarning{Making the problem clear a few lines before makes easier to justify how this solution will address it} which requires less computations\fxnote{Needs citation} and by implementing the algorithm in a low level language.\fxnote{Needs citation} \fxnote{Add introduction section to existing solutions with references. Discuss the issues.}\\

\fxwarning{Then, make the connection to your setting (low-power construction device) and the direct application. How a deep-learning model will help this constraint device?}

%%%%% Colberg %%%%%
% To be added:
% We broadcast limited amount of data as compared to 36 datapoints
% Low power computation device - Heavy algoritms (NN) takes more computational power
% More complex model than threshold but still more simple than current methods using both gyroscopes, accelerometers, image recogniction and/or complex models like RNN, DNN, CNN or a combination of them. We use only one data collection source and a simple algorithm

\fxwarning{Connect to the previous idea. For example, the device is capturing time series, so...}
Time series analysis is used in a variety of domains such as Human activity recognition \cite{HumanActivityrecognitionAccelerometer}, construction activity recognition \cite{ConstructionRecognitionFractionalRandomForest}\cite{timeseriesDataAugmentationConstruction}, weather forecasting \cite{weatherForecastTimeSeries} etc. 
The application of time series analysis in the construction industry has shown promising results on medium sized equiptment such as escavators, lifts and trucks \cite{timeseriesDataAugmentationConstruction, constructionRecognitionMobileSensors,ConstructionRecognitionFractionalRandomForest}, \fxwarning{Why can be promising? Better precision/detail?} however research involving small size equipment such as drills, generators and minigravers is sparse\fxnote{Not sure how to write this, but i want to say that there is room for research in this area}. \fxwarning{Maybe something like: "there is limited research in the implementation of deep learning models in hardware-specific devices (because processing constraints?), most of them are focus on..." and look for some citation. }
Of particular interest is works related to accelerometer data \cite{HumanActivityrecognitionAccelerometer,timeseriesDataAugmentationConstruction} as the small size and low weight of the accelerometer makes it easy to fit on smaller device. \fxwarning{How this connect to the previous idea, why accelerometer data? Highly available sensors?}
Measuring the movement of a given device and classifying it can help identify how it is treated and identify specific patterns that leads to destruction or misuse. \fxwarning{It will fit better a couple of lines before.}
Identifying such patterns allows the industry to minimize deteriating behavior, meaning less ressources is spent on checking, fixing or buying new equipment. \fxwarning{It will fit better a couple of lines before.}
If an accelerometer is fitted to equipment emitting $CO_2$, such as a generator, measuring its activity can help reduce the time it is in use when it is not needed \fxwarning{It could use to connect the accelerometer with the devices.}.\\ 
