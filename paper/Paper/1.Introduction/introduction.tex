In the 1990s scientists started using machine learning to analyze large amounts of data and draw conclusions from the results \cite{marr2016short}.
One of the issues with this, is that processing data at this scale requires a lot of computational power \cite{garcia2019estimation}, which means that devices with lower computational power will struggle to process the data in a timely manner.
This issue is amplified by the fact that high-level languages, which are large and complex, provide the best support for machine learning algorithms due to their large number of libraries \cite{top5mllangs}.
This could be a problem for low consumption devices, because computing power and battery life might be limited. This creates the question: 
\begin{center}
\textbf{Is it possible to run an embedded Support Vector Machine (SVM) on a low consumption device?}
\end{center}
If you have a low consumption device, which has low computing power and limited battery capacity, running machine learning inference can be a demanding task on the device's hardware \cite{batterylifewearablesensors}. Therefore, it is interesting to investigate, if a machine learning algorithm can be embedded on a low consumption device, in order to reduce strain on the device's hardware.\\

\fxwarning{Then, make the connection to your setting (low-power construction device) and the direct application. How a deep-learning model will help this constraint device?}

%%%%% Colberg %%%%%
% To be added:
% We broadcast limited amount of data as compared to 36 data points
% Low power computation device - Heavy algorithms (NN) takes more computational power
% More complex model than threshold but still more simple than current methods using both gyroscopes, accelerometers, image recognition and/or complex models like RNN, DNN, CNN or a combination of them. We use only one data collection source and a simple algorithm

\fxwarning{Connect to the previous idea. For example, the device is capturing time series, so...}
Time series analysis is used in a variety of domains such as Human activity recognition \cite{HumanActivityrecognitionAccelerometer}, construction activity recognition \cite{ConstructionRecognitionFractionalRandomForest}\cite{timeseriesDataAugmentationConstruction}, weather forecasting \cite{weatherForecastTimeSeries} etc. 
The application of time series analysis in the construction industry has shown promising results on medium sized equipment such as escavators, lifts and trucks reaching above 96\% accuracy \cite{timeseriesDataAugmentationConstruction, constructionRecognitionMobileSensors,ConstructionRecognitionFractionalRandomForest} using augmentation, long short-term memory (LSTM) \cite{timeseriesDataAugmentationConstruction} support vector machines \cite{constructionRecognitionMobileSensors} and fractional random forest \cite{ConstructionRecognitionFractionalRandomForest}, however research in deep learning involving is limited in hardware-specific devices, because of the memory and proccesing demands \cite{deepLearningLowConsumptionStateOfTheArt} , most of them are focus on..." and look for some citation.
Of particular interest is works related to accelerometer data \cite{HumanActivityrecognitionAccelerometer,timeseriesDataAugmentationConstruction} as the small size and low weight of the accelerometer makes it easy to fit on smaller device. \fxwarning{How this connect to the previous idea, why accelerometer data? Highly available sensors?}
Measuring the movement of a given device and classifying it can help identify how it is treated and identify specific patterns that leads to destruction or misuse. \fxwarning{It will fit better a couple of lines before.}
Identifying such patterns allows the industry to minimize deteriating behavior, meaning less ressources is spent on checking, fixing or buying new equipment. \fxwarning{It will fit better a couple of lines before.}
If an accelerometer is fitted to equipment emitting $CO_2$, such as a generator, measuring its activity can help reduce the time it is in use when it is not needed \fxwarning{It could use to connect the accelerometer with the devices.}.\\ 
