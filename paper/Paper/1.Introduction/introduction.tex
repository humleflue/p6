Machine learning has many applications. 
Since its discovery in the 1950s, machine learning has been used to solve complex problems in various domains \fxnote{This is very broad. Other articles starts closer to their domain. This is either done by introducing the application domain (construction industry) or time series forecasting on accelerometer data (the reasoning or beginning behind. Also if none or insufficient research has been done it is mentioned in the beginning)}\fxnote{Needs citation}. 
In the 1990s the focus of machine learning shifted from a knowledge-driven approach to a data-driven approach \cite{marr2016short}\fxnote{See first fxnote}. 
This is when scientists started using machine learning to analyze large amounts of data and draw conclusions from the results \cite{marr2016short}. 
One of the problems with this, is that it takes a lot of computing power to process data at this scale. 
In present time, this problem is amplified by the fact, that Python is the programming language, which has the best support for machine learning algorithms, through its large amount of libraries and user friendly syntax\fxnote{Needs citation. Seems like a holdning}. 
This creates a problem when it comes to low consumption devices. 
If you have a low consumption device, which has low computing power and limited battery capacity, running machine learning inference can be a demanding task on the device's hardware\fxnote{This should be the first sentence with some rewriting imo. Quick introduction to low consumption devices and their limitations - hence our }. 
This can be combatted by using simple algorithms, which requires less computations\fxnote{Needs citation} and by implementing the algorithm in a low level language.\fxnote{Needs citation} \fxnote{Add introduction section to existing solutions with references. Discuss the issues.}\\


%%%%% Colberg %%%%%
% To be added:
% We broadcast limited amount of data as compared to 36 datapoints
% Low power computation device - Heavy algoritms (NN) takes more computational power
% More complex model than threshold but still more simple than current methods using both gyroscopes, accelerometers, image recogniction and/or complex models like RNN, DNN, CNN or a combination of them. We use only one data collection source and a simple algorithm

Time series analysis is used in a variety of domains such as Human activity recognition \cite{HumanActivityrecognitionAccelerometer}, construction activity recognition \cite{ConstructionRecognitionFractionalRandomForest}\cite{timeseriesDataAugmentationConstruction}, weather forecasting \cite{weatherForecastTimeSeries} etc. 
The application of time series analysis in the construction industry has shown promising results on medium sized equiptment such as escavators, lifts and trucks \cite{timeseriesDataAugmentationConstruction, constructionRecognitionMobileSensors,ConstructionRecognitionFractionalRandomForest}, however research involving small size equipment such as drills, generators and minigravers is sparse\fxnote{Not sure how to write this, but i want to say that there is room for research in this area}. 
Of particular interest is works related to accelerometer data \cite{HumanActivityrecognitionAccelerometer,timeseriesDataAugmentationConstruction} as the small size and low weight of the accelerometer makes it easy to fit on smaller device. 
Measuring the movement of a given device and classifying it can help identify how it is treated and identify specific patterns that leads to destruction or misuse. 
Identifying such patterns allows the industry to minimize deteriating behavior, meaning less ressources is spent on checking, fixing or buying new equipment. 
If an accelerometer is fitted to equipment emitting $CO_2$, such as a generator, measuring its activity can help reduce the time it is in use when it is not needed.\\
