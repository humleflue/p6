In the 1990s scientists started using machine learning to analyze large amounts of data and draw conclusions from the results \cite{marr2016short}.
One of the issues with this, is that processing data at this scale requires a lot of computational power \cite{garcia2019estimation}, which means that devices with lower computational power will struggle to process the data in a timely manner.
This issue is amplified by the fact that high-level languages, which are large and complex, provide the best support for machine learning algorithms due to their large number of libraries \cite{top5mllangs}. \fxwarning{Until here, the idea seems like: data-driven requires a lot of computation + Python is big, but the problem definition is not explicit. Try to detail it before talking about the low-consumption devices }
This creates a problem when it comes to low consumption devices. 
If you have a low consumption device, which has low computing power and limited battery capacity, running machine learning inference can be a demanding task on the device's hardware \fxwarning{Here, you can make the constrast to the problem definition, why something big does not run properly in these devices?} \fxnote{This should be the first sentence with some rewriting imo. Quick introduction to low consumption devices and their limitations - hence our }. 
This can be combatted by using simple algorithms, \fxwarning{Making the problem clear a few lines before makes easier to justify how this solution will address it} which requires less computations\fxnote{Needs citation} and by implementing the algorithm in a low level language.\fxnote{Needs citation} \fxnote{Add introduction section to existing solutions with references. Discuss the issues.}\\

\fxwarning{Then, make the connection to your setting (low-power construction device) and the direct application. How a deep-learning model will help this constraint device?}

%%%%% Colberg %%%%%
% To be added:
% We broadcast limited amount of data as compared to 36 datapoints
% Low power computation device - Heavy algoritms (NN) takes more computational power
% More complex model than threshold but still more simple than current methods using both gyroscopes, accelerometers, image recogniction and/or complex models like RNN, DNN, CNN or a combination of them. We use only one data collection source and a simple algorithm

\fxwarning{Connect to the previous idea. For example, the device is capturing time series, so...}
Time series analysis is used in a variety of domains such as Human activity recognition \cite{HumanActivityrecognitionAccelerometer}, construction activity recognition \cite{ConstructionRecognitionFractionalRandomForest}\cite{timeseriesDataAugmentationConstruction}, weather forecasting \cite{weatherForecastTimeSeries} etc. 
The application of time series analysis in the construction industry has shown promising results on medium sized equiptment such as escavators, lifts and trucks \cite{timeseriesDataAugmentationConstruction, constructionRecognitionMobileSensors,ConstructionRecognitionFractionalRandomForest}, \fxwarning{Why can be promising? Better precision/detail?} however research involving small size equipment such as drills, generators and minigravers is sparse\fxnote{Not sure how to write this, but i want to say that there is room for research in this area}. \fxwarning{Maybe something like: "there is limited research in the implementation of deep learning models in hardware-specific devices (because processing constraints?), most of them are focus on..." and look for some citation. }
Of particular interest is works related to accelerometer data \cite{HumanActivityrecognitionAccelerometer,timeseriesDataAugmentationConstruction} as the small size and low weight of the accelerometer makes it easy to fit on smaller device. \fxwarning{How this connect to the previous idea, why accelerometer data? Highly available sensors?}
Measuring the movement of a given device and classifying it can help identify how it is treated and identify specific patterns that leads to destruction or misuse. \fxwarning{It will fit better a couple of lines before.}
Identifying such patterns allows the industry to minimize deteriating behavior, meaning less ressources is spent on checking, fixing or buying new equipment. \fxwarning{It will fit better a couple of lines before.}
If an accelerometer is fitted to equipment emitting $CO_2$, such as a generator, measuring its activity can help reduce the time it is in use when it is not needed \fxwarning{It could use to connect the accelerometer with the devices.}.\\ 
