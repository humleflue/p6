In the 1990s scientists started using machine learning to analyze large amounts of data and draw conclusions from the results \cite{marr2016short}.
One of the issues with this, is that processing data at this scale requires a lot of computational power \cite{garcia2019estimation}, which means that devices with lower computational power will struggle to process the data in a timely manner.
This issue is amplified by the fact that high-level languages, which are often large and complex \fxnote{missing cite? (this is a statement from David)}, provide the best support for machine learning algorithms due to their large number of libraries \cite{top5mllangs}.
This could be a problem for low consumption devices, because computing power and battery life might be limited. As a result, the following question arises: 
\begin{center}
\textbf{Is it possible to run an embedded Support Vector Machine (SVM) on a low consumption device gaining a higher accuracy than a threshold?}
\end{center}
If you have a low consumption device, which has low computing power and limited battery capacity, running machine learning inference can be a demanding task on the device's hardware \cite{batterylifewearablesensors}. Therefore, it is interesting to investigate, if a machine learning algorithm can be embedded on a low consumption device, in order to reduce strain on the device's hardware.\\

% \fxwarning{Then, make the connection to your setting (low-power construction device) and the direct application. How a deep-learning model will help this constraint device?} % LASSE: I don't know if I have actually resolved this, so I will leave it here for now. Please remove it if you think we have.

%%%%% Colberg %%%%%
% To be added:
% We broadcast limited amount of data as compared to 36 data points
% More complex model than threshold but still more simple than current methods using both gyroscopes, accelerometers, image recognition and/or complex models like RNN, DNN, CNN or a combination of them. We use only one data collection source and a simple algorithm

The low consumption device, which we will be testing is using time series data, and it has applications within the construction industry.\\
Time series analysis is used in a variety of domains such as Human Activity Recognition (HAR) \cite{HumanActivityrecognitionAccelerometer}, construction activity recognition \cite{ConstructionRecognitionFractionalRandomForest}\cite{timeseriesDataAugmentationConstruction} and weather forecasting \cite{weatherForecastTimeSeries}.
The application of time series analysis in the construction industry has shown promising results on medium sized equipment such as excavators, lifts and trucks reaching accuracies above 96\% \cite{timeseriesDataAugmentationConstruction, constructionRecognitionMobileSensors,ConstructionRecognitionFractionalRandomForest} using augmentation, long short-term memory (LSTM) \cite{timeseriesDataAugmentationConstruction} support vector machines \cite{constructionRecognitionMobileSensors} and fractional random forest \cite{ConstructionRecognitionFractionalRandomForest}, however research in deep learning is limited in hardware-specific devices, because of the memory and processing demands \cite{deepLearningLowConsumptionStateOfTheArt}. Most of them are focus on heavy computation running on Graphics Processing Units (GPUs) with Compute Unified Device Architecture (CUDA) processors and therefore they cannot be directly applied on mobile and embedded devices \cite{deepLearningLowConsumptionStateOfTheArt}.
Measuring the movement of a small device such as a drill and classifying it can help identify how it is treated and identify specific patterns that leads to destruction or misuse. Identifying such patterns allows the industry to minimize deteriorating behavior, meaning that less resources are spent on checking, fixing and buying new equipment.\\ 
Other domains expand research of deep learning to low consumption devices such as HAR \cite{HumanActivityrecognitionAccelerometer, HARsignalprocessing, hybridHARSVMCNN} as these run deep learning on mobile or embedded devices. They all use accelerometers for data collection, which is advantageous because the accelerometer's compact size and low weight make it easier to fit on smaller devices like mobile or embedded devices, and it produces basic data with few features to evaluate, reducing processing need.

As deep learning algorithms are often big and complex \fxnote{missing cite (this statement is from David)}, we will be taking a closer look at a simpler and more efficient algorithm \fxnote{missing cite (this statement is from David)}; a Support Vector Machine (SVM). This is chosen as we want a robust algorithm which runs fast and with low complexity.