\section{Related work}
% NEEDS CITATION. This is done after introduction is merged with
\subsection{Activity recognition}
Acivity recognition is used in a variety of domains, most noticable in the area of human activity recognition where different applications such as recognising gestures and motions (running, walking, falling)\fxnote{insert citation once introduction have been merged with main}. They offer different approaches varying in the type of activity recognition method. The three major activity recognition methods is kinetic-based, computer vision-based and audio-based \fxnote{cite this https://ascelibrary-org.zorac.aub.aau.dk/doi/epdf/10.1061/\%28ASCE\%29CO.1943-7862.0001843} with kinetic-based being the one applied for our algorithm. Current solutions regardless of the method applies neural network (NN) algorithms in different forms as deep neural networks (DNN), convolutional neural networks (CNN) and reccurent neural networks (RNN) or a combination of if with simpler models such as random forest (RF) and SVM \fxnote{cite this https://ieeexplore-ieee-org.zorac.aub.aau.dk/stamp/stamp.jsp?tp=&arnumber=9425332&tag=1}. The proposed solutions typpically uses a mix of accelerometer and gyroscope data at least, but some also use image recognition for hand gestures by means of a CNN. Such complex and load intensive algorithm is not satisfiable due to the low compuation needed, however in combination with research done within the construction industry \fxnote{cite this https://ascelibrary-org.zorac.aub.aau.dk/doi/epdf/10.1061/\%28ASCE\%29CO.1943-7862.0001843} where research is mainly focused on medium sized equipment we find a suitable solution for small size equipment by applying methods from human activity recognition alongside the methods within the construction activity recognition domain.
\subsection{Construction activity recognition}
Current research of activity recognition in the construction industry revolves around medium and large sized equipment \fxnote{cite https://ascelibrary-org.zorac.aub.aau.dk/doi/epdf/10.1061/\%28ASCE\%29CO.1943-7862.0001843 https://reader.elsevier.com/reader/sd/pii/S0926580517305149?token=CC48E75D26781C7C65E88FDD928B90B85C0ED574C461562F47A52022B1F79B139B0AD15FB32FC2CF1A17B4619DC4B1EB&originRegion=eu-west-1&originCreation=20220404121314 https://reader.elsevier.com/reader/sd/pii/S1474034619300886?token=27A6D898D6225A9098285365360FF5A19A28B74D27E19CB683AEC0B58CB260C47D56E656F68305ADF0D45262C188C29D&originRegion=eu-west-1&originCreation=20220404071807}. They make use of NN's, RF's and SVM's working on a combination of accelerometer data, gyroscope data and image recognition. A state of the art review showed that within kinetic-based methods a RNN working on 6 motions given internal data from the joystick was the best solution with a 96\% accuracy \fxnote{cite https://ascelibrary-org.zorac.aub.aau.dk/doi/epdf/10.1061/\%28ASCE\%29CO.1943-7862.0001843} and upwards of 99\% accuracy when meassuring a workes activity in a lab environment using a SVM and an accelerometer. As the SVM is a computational light model compared to NN \fxnote{cite https://www.researchgate.net/profile/J-E-T-Akinsola/publication/318338750_Supervised_Machine_Learning_Algorithms_Classification_and_Comparison/links/596481dd0f7e9b819497e265/Supervised-Machine-Learning-Algorithms-Classification-and-Comparison.pdf} it is favourable for implementation on the \kin.
\subsection{Radial basis function}
relevant to write anything about this?
