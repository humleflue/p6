\section{Related work}
% NEEDS CITATION. This is done after introduction is merged with
\fxwarning{Introduce the section with few lines. It is not clear why it start with activity recognition}
\subsection{Activity recognition}
Activity recognition is used in a variety of domains, most noticeable in the area of human activity recognition where different applications such as recognizing gestures and motions (running, walking, falling) \fxwarning{Requires citation}\fxnote{insert citation once introduction have been merged with main}. 
They offer different approaches varying in the type of activity recognition method. \fxwarning{Which approaches?}
The three major activity recognition methods is kinetic-based, computer vision-based and audio-based \fxwarning{You can elaborate in a couple of lines } \cite{soaWorkersAndEquipment} with kinetic-based being the one applied for our algorithm. \fxwarning{Why is your algorithm kinetic-based?}
\fxwarning{Make the connection. For instance, "for recognizing activities, X and Y methods are used, but other solutions can be applied, such as..." }
Current solutions regardless of the method applies neural network (NN) \fxwarning{Citation} algorithms in different forms as deep neural networks (DNN) \fxwarning{Citation}, Convolutional Neural Networks (CNN) \fxwarning{Citation} and recurrent neural networks (RNN) \fxwarning{Citation} or a combination of if with simpler models such as random forest (RF) \fxwarning{Citation} and SVM \cite{hybridHARSVMCNN}. 
The proposed solutions typically uses a mix of accelerometer and gyroscope data at least \fxwarning{Citation}, but some also use image recognition for hand gestures by means of a CNN. \fxwarning{Citation. Related to others than kinetic?}. 
Such complex and load intensive algorithm \fxwarning{Try to connect to the previous idea} is not satisfiable due to the low computation needed, however in combination with research done within the construction \fxwarning{Both concepts seems unrelated, implictly is the monitoring device} industry \cite{soaWorkersAndEquipment} where research is mainly focused on medium sized equipment \fxwarning{Do they make classification? You can briefly mention their idea} we find \fxwarning{Propose?} a suitable solution \fxwarning{How is suitable? What connect both papers/domain?} for small size equipment by applying methods from human activity recognition alongside the methods within the construction activity recognition domain.

\subsection{Construction activity recognition}
Current research of activity recognition in the construction industry revolves around medium and large sized equipment \cite{soaWorkersAndEquipment, activityAudioSVM, timeseriesDataAugmentationConstruction}
They make use of NN's, RF's and SVM's working on a combination of accelerometer data, gyroscope data and image recognition. \fxwarning{Is there a drawback for using the settings? Many data sources, efficient methods, small equipment is missed? }
A state of the art review showed that within kinetic-based methods a RNN working on 6 motions given internal data from the joystick was the best solution with a 96\% accuracy \cite{soaWorkersAndEquipment} and upwards of 99\% accuracy when measuring a works activity in a lab environment using a SVM and an accelerometer. \fxwarning{You can more general since that work is a review of methods. For instance, what are they main findings?}
As the SVM is a computational light model compared to NN \cite{comparisonMLAlgorithms} it is favorable for implementation on the \kin. \fxwarning{Why is favorable?}


\subsection{Radial basis function}
\fxwarning{It is not related to this section}
% relevant to write anything about this?
